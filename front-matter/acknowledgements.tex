Thank you to the Department of Computational Medicine \& Bioinformatics for
funding my research and training via the NIH Training Program in Bioinformatics
(T32 GM070449).

Thank you to all the people who supported me in this journey and helped shape me
to become who I am:

My family: Mom, Dad, Erika, and Katie, and my partner David.
Your support means everything to me.
My extended family for not disowning me for attending the University of Michigan
despite your other allegiances.
All my friends from this chapter of life and the chapters preceding it.

Kristi Janson, my high school teacher who taught the biomedical sciences
courses that made me want to become a scientist.
Rachelle Sells Galvin and the scientists at Eli Lilly \& Company
who graciously organized a day for me to shadow them,
where they shared their passion for science and gave me great advice for
pursuing research experiences in college.

My undergraduate research mentors at the University of Kentucky:
Jerzy Jaromczyk, Neil Moore, Dave Weisrock, and Hunter Moseley.
Hunter Moseley spent many hours critiquing my code and cultivated my programming
skills from crawling to running.

All those involved in The Carpentries instance at the University of Michigan who
contributed to the curriculum, taught in the pilot workshops, and continue to
empower others to learn how to code.

Past and present members of our chapter of Girls Who Code at U-M DCMB:
Brooke Wolford and Zena Lapp, who co-founded the chapter;
Marlena Duda, who spearheaded the effort to develop our custom curriculum; and
Audrey Drotos and Hayley Falk, who are passing on the torch to the next
generation of graduate students to sustain this club for many years to come.
DCMB, for unwaivering support of the club since the beginning, especially our
faculty co-sponsors Maureen Sartor and Cristina Mitrea.
Thanks to all the students who participated and tried something new.

My committee members Greg Dick, Peter Freddolino, Jenna Wiens, and Vince Young
for your helpful insights.
Krishna Rao for helping me understand clinical guidelines for diagnosing
and treating \textit{C. difficile} infections.

Members of the Schloss Lab past and present.
You are all wonderful people and you made graduate school a great experience.

My advisor and mentor, Pat. You are the best!
