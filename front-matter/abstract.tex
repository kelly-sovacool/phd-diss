The human gut microbiome plays an important role in maintaining health.
Changes in the taxonomic composition and metabolic activity of the gut
microbiota have been implicated in numerous diseases including colorectal
cancer, \textit{Clostridioides difficile} infection (CDI), irritable bowel disease,
and others.
Thus, the gut microbiome is a promising source of biomarkers for disease
diagnosis and prediction.
Machine learning (ML) approaches can integrate big data to gain insight into
patterns that aren't interrogable by traditional statistics.
Here, we introduce a new algorithm to improve capabilities of using ML for
microbiome research,
apply ML to predict severe outcomes of CDI,
and introduce resources that empower data scientists from a range of skill
levels to take programming skills from basic through being able to apply data
science responsibly to investigate their own research questions.

Assigning amplicon sequences to operational taxonomic units (OTUs) is an
important step in characterizing microbial communities across large datasets.
However, a gap in existing OTU assignment methods inhibited the ability of
researchers to incorporate new samples to previously clustered datasets without
clustering all sequences again, such as when comparing across datasets or
deploying machine learning models.
To provide an efficient method to fit sequences to existing OTUs while
maintaining high OTU quality, we developed the OptiFit algorithm, an improved
implementation of reference-based clustering.
Our benchmarks revealed that OptiFit produces similar quality OTUs as a gold
standard method yet at faster speeds.
Thus, OptiFit provides a suitable option for users requiring consistent OTU
assignments at the same quality afforded by \textit{de novo} clustering methods.

CDI can lead to adverse outcomes including ICU admission, colectomy, and death,
with half a million cases annually in the United States. The composition of the
gut microbiome plays an important role in determining colonization resistance
and clearance upon exposure to \textit{C. difficile}.
We investigated whether ML models trained on OTUs from patient stool samples on
the day of CDI diagnosis could predict which CDI cases led to severe outcomes.
We trained ML models to predict CDI severity on OTU relative abundances
according to four different severity definitions described by prior studies.
The models performed best when predicting pragmatic severity, a composite
definition based on the occurrence of complications due to any cause or
confirmed as CDI-attributable via chart review when possible.
Our results suggest that while chart review is valuable to verify the cause of
complications, including as many samples as possible is indispensable for
training performant models on imbalanced datasets.
We evaluated the potential clinical value of the OTU-based models and found
similar performance compared to prior models based on Electronic Health Records,
although further work is needed to determine the feasibility of deploying such
models in clinical practice.
These OTU-based models represents a step toward the goal of deploying ML to
inform clinical decisions and ultimately improve CDI outcomes.

Bioinformatics is a kind of data science, an interdisciplinary field integrating
computer science, statistics, and domain knowledge.
Novice researchers frequently have domain knowledge, but lack other skills necessary to
apply data science to their datasets while adhering to best practices in
reproducibility.
We developed three resources to help democratize data science:
a curriculum for high school students to learn the basics of Python for data science,
an introductory curriculum for researchers to learn best practices in coding for
reproducible science,
and an R package implementing a framework of best practices in ML that allows
novices to apply ML responsibly while being malleable for advanced users.
These resources cover a breadth of audience skill levels to help fill several
gaps in existing tools for reproducible data science.